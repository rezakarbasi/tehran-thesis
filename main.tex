% !TEX TS-program = xelatex
% Command for running this example (needs latexmkrc file):
%    latexmk -bibtex -pdf main.tex

%	نمونه پایان‌نامه آماده شده با استفاده از کلاس tehran-thesis، نگارش 1
%	سینا ممکن، دانشگاه تهران 
%	https://github.com/sinamomken/tehran-thesis
%	گروه پارسی‌لاتک
%	http://www.parsilatex.com
%	این نسخه، بر اساس نسخه‌ 0.1 از کلاس IUST-Thesis آقای محمود امین‌طوسی آماده شده است.
%        http://profsite.sttu.ac.ir/mamintoosi

%----------------------------------------------------------------------------------------------
% اگر قصد نوشتن پروژه کارشناسی را دارید، در خط زیر به جای msc، کلمه bsc و اگر قصد نوشتن رساله دکترا را دارید، کلمه phd را قرار دهید. کلیه تنظیمات لازم، به طور خودکار، اعمال می‌شود.

% اگر مایلید پایان‌نامه شما دورو باشد به جای oneside در خط زیر از twoside استفاده کنید.

% برای حاشیه‌نویسی و کم کردن صفحات ابتدایی، گزینه draft را وارد و برای نسخه نهایی آن را حذف کنید.

% برای استفاده از قلم‌های سری IR Series گزینه irfonts را وارد و برای استفاده از قلم‌های X Series 2 آن را حذف کنید.

\documentclass[
twoside
% ,openany
,msc
,irfonts
 ,draft
]{./tex/tehran-thesis}

% فایل commands.tex را مطالعه کنید؛ چون دستورات مربوط به فراخوانی بسته‌ها، فونت و دستورات خاص در این فایل قرار دارد.
% در این فایل، دستورها و تنظیمات مورد نیاز، آورده شده است.
%-------------------------------------------------------------------------------------------------------------------
% دستوراتی که پوشه پیش‌فرض زیرفایل‌های tex را مشخص می‌کند.
%\makeatletter
%\def\input@path{{./tex/}}
%\makeatother
% در ورژن جدید زی‌پرشین برای تایپ متن‌های ریاضی، این سه بسته، حتماً باید فراخوانی شود
\usepackage{amsthm,amssymb,amsmath}
% بسته‌ای برای تنطیم حاشیه‌های بالا، پایین، چپ و راست صفحه
\usepackage[a4paper, top=40mm, bottom=40mm, outer=25mm, inner=35mm]{geometry}
% بسته‌‌ای برای ظاهر شدن شکل‌ها و تعیین آدرس تصاویر
\usepackage[final]{graphicx}
\graphicspath{{./img/}}
% بسته‌های مورد نیاز برای نوشتن کدها، رنگ‌آمیزی آنها و تعیین پوشهٔ کدها
\usepackage[final]{listings}
\usepackage[usenames,dvipsnames,svgnames,table]{xcolor}
\lstset{inputpath=./code/}
% بسته‌ای برای رسم کادر
\usepackage{framed} 
% بسته‌‌ای برای چاپ شدن خودکار تعداد صفحات در صفحه «معرفی پایان‌نامه»
\usepackage{lastpage}
% بسته‌ٔ لازم برای: ۱. تغییر شماره‌گذاری صفحات پیوست. ۲. تصحیح باگ آدرس وب حاوی '%' در مراجع
\usepackage{etoolbox}

%%%%%%%%%%%%%%%%%%%%%%%%%%%%%%%%%%%%
%%% دستورات وابسته به استیل مراجع:
%% اگر از استیل‌های natbib (plainnat-fa، asa-fa، chicago-fa) استفاده می‌کنید، خط زیر را فعال و بعدی‌اش را غیرفعال کنید.
%\usepackage{natbib}
%\newcommand{\citelatin}[1]{\cite{#1}\LTRfootnote{\citeauthor*{#1}}}
%\newcommand{\citeplatin}[1]{\citep{#1}\LTRfootnote{\citeauthor*{#1}}}
%% اگر از سایر استیل‌ها استفاده می‌کنید، خط بالا را غیرفعال و خط‌های زیر را فعال کنید.
\let\citep\cite
\let\citelatin\cite
\let\citeplatin\cite
%%%%%%%%%%%%
% بررسی حالت پیش نویس
\usepackage{ifdraft}
\ifdraft
{%
	% بسته‌ٔ ایجاد لینک‌های رنگی با امکان جهش
	\usepackage[unicode=true,pagebackref=true,
colorlinks,linkcolor=blue,citecolor=blue,final]{hyperref}
	%\usepackage{todonotes}
	\usepackage[firstpage]{draftwatermark}
	\SetWatermarkText{\ \ \ پیش‌نویس}
	\SetWatermarkScale{1.2}
}
{ 
	\usepackage[pagebackref=false]{hyperref}
	%\usepackage[disable]{todonotes} % final without TODOs
}

\usepackage[obeyDraft]{todonotes}
\setlength{\marginparwidth}{2cm}

%%%%%%%%%%%%
%%% تصحیح باگ: اگر در مراجع، آدرس وب حاوی '%' بوده و pagebackref فعال باشد، دستورات زیر باید بیایند:
%% برای استیل‌های natbib مثل plainnat-fa، asa-fa، chicago-fa
\makeatletter
\let\ORIG@BR@@lbibitem\BR@@lbibitem
\apptocmd\ORIG@BR@@lbibitem{\endgroup}{}{}
\def\BR@@lbibitem{\begingroup\catcode`\%=12 \ORIG@BR@@lbibitem}
\makeatother
%% برای سایر استیل‌ها
\makeatletter
\let\ORIG@BR@@bibitem\BR@@bibitem
\apptocmd\ORIG@BR@@bibitem{\endgroup}{}{}
\def\BR@@bibitem{\begingroup\catcode`\%=12 \ORIG@BR@@bibitem}
\makeatother
%%%%%%%%%%%%%%%%%%%%%%%%%%%%%%%%%%%%

% بسته‌ لازم برای تنظیم سربرگ‌ها
\usepackage{fancyhdr}
%\usepackage{enumitem}
\usepackage{setspace}
% بسته‌های لازم برای نوشتن الگوریتم
\usepackage{algorithm}
\usepackage{algorithmic}
% بسته‌های لازم برای رسم بهتر جداول
\usepackage{tabulary}
\usepackage{tabularx}
\usepackage{rotating}
% بسته‌های لازم برای رسم تنظیم بهتر شکل‌ها و زیرشکل‌ها
\usepackage[export]{adjustbox}
\usepackage{subfig}
\usepackage[subfigure]{tocloft}
% بسته‌ای برای رسم نمودارها و نیز صفحه مالکیت اثر
\usepackage{tikz}
% بسته‌ای برای ظاهر شدن «مراجع» و «نمایه» در فهرست مطالب
\usepackage[nottoc]{tocbibind}
% دستورات مربوط به ایجاد نمایه
\usepackage{makeidx}
\makeindex
%%% بسته ایجاد واژه‌نامه با xindy
\usepackage[xindy,toc,acronym,nonumberlist=true]{glossaries}

% بسته‌ای برای افزودن تورفتگی به ابتدای اولین پاراگراف هر بخش
\usepackage{indentfirst}

% بسته زیر باگ ناشی از فراخوانی بسته‌های زیاد را برطرف می‌کند.
\usepackage{morewrites}
%%%%%%%%%%%%%%%%%%%%%%%%%%
% فراخوانی بسته زی‌پرشین (باید آخرین بسته باشد)
\usepackage[extrafootnotefeatures, localise=on, displaymathdigits=persian]{xepersian}




\makeatletter
% تعریف قلم فارسی و انگلیسی و مکان قلم‌ها
\if@irfonts
\settextfont[Path={./font/}, BoldFont={IRLotusICEE_Bold.ttf}, BoldItalicFont={IRLotusICEE_BoldIranic.ttf}, ItalicFont={IRLotusICEE_Iranic.ttf},Scale=1.2]{IRLotusICEE.ttf}
% LiberationSerif or FreeSerif as free equivalents of Times New Roman
\setlatintextfont[Path={./font/}, BoldFont={LiberationSerif-Bold.ttf}, BoldItalicFont={LiberationSerif-BoldItalic.ttf}, ItalicFont={LiberationSerif-Italic.ttf},Scale=1]{LiberationSerif-Regular.ttf}

%Reza Changed
\ExplSyntaxOn \cs_set_eq:NN \etex_iffontchar:D \tex_iffontchar:D \ExplSyntaxOff

% چنانچه می‌خواهید اعداد در فرمول‌ها، انگلیسی باشد، خط زیر را غیرفعال کنید
% و گزینهٔ displaymathdigits=persian را از خط ۱۰۹ حذف کنید.
\setdigitfont[Path={./font/}, Scale=1.2]{IRLotusICEE.ttf}
% تعریف قلم‌های فارسی و انگلیسی اضافی برای استفاده در بعضی از قسمت‌های متن
\setiranicfont[Path={./font/}, Scale=1.3]{IRLotusICEE_Iranic.ttf}				% ایرانیک، خوابیده به چپ
\setmathsfdigitfont[Path={./font/}]{IRTitr.ttf}
\defpersianfont\titlefont[Path={./font/}, Scale=1]{IRTitr.ttf}
% برای تعریف یک قلم خاص عنوان لاتین، خط بعد را فعال و ویرایش کنید و خط بعد از آن را غیرفعال کنید.
% \deflatinfont\latintitlefont[Scale=1]{LiberationSerif}
\font\latintitlefont=cmssbx10 scaled 2300 %cmssbx10 scaled 2300
\else
\settextfont{XB Niloofar}
\setlatintextfont{Junicode}
% چنانچه می‌خواهید اعداد در فرمول‌ها، انگلیسی باشد، خط زیر را غیرفعال کنید
% و گزینهٔ displaymathdigits=persian را از خط ۱۰۹ حذف کنید.
%Reza Changed
\ExplSyntaxOn \cs_set_eq:NN \etex_iffontchar:D \tex_iffontchar:D \ExplSyntaxOff

\setdigitfont{XB Niloofar}
% تعریف قلم‌های فارسی و انگلیسی اضافی برای استفاده در بعضی از قسمت‌های متن
% \setmathsfdigitfont{XB Titre}
\defpersianfont\titlefont{XB Titre}
\deflatinfont\latintitlefont[Scale=1.1]{Junicode}
\fi
\makeatother

% برای استفاده از قلم نستعلیق خط بعد را فعال کنید.
% \defpersianfont\nastaliq[Scale=1.2]{IranNastaliq}


%%%%%%%%%%%%%%%%%%%%%%%%%%
% راستچین شدن todonotes
\presetkeys{todonotes}{align=right,textdirection=righttoleft}{}
\makeatletter
\providecommand\@dotsep{5}
\def\listtodoname{فهرست کارهای باقیمانده}
\def\listoftodos{\noindent{\Large\vspace{10mm}\textbf{\listtodoname}}\@starttoc{tdo}}
\renewcommand{\@todonotes@MissingFigureText}{شکل}
\renewcommand{\@todonotes@MissingFigureUp}{شکل}
\renewcommand{\@todonotes@MissingFigureDown}{جاافتاده}
\makeatother
% دستوری برای حذف کلمه «چکیده»
\renewcommand{\abstractname}{}
% دستوری برای حذف کلمه «abstract»
%\renewcommand{\latinabstract}{}
% دستوری برای تغییر نام کلمه «اثبات» به «برهان»
\renewcommand\proofname{\textbf{برهان}}
% دستوری برای تغییر نام کلمه «کتاب‌نامه» به «مراجع»
\renewcommand{\bibname}{مراجع}
% دستوری برای تعریف واژه‌نامه انگلیسی به فارسی
\newcommand\persiangloss[2]{#1\dotfill\lr{#2}\\}
% دستوری برای تعریف واژه‌نامه فارسی به انگلیسی 
\newcommand\englishgloss[2]{#2\dotfill\lr{#1}\\}
% تعریف دستور جدید «\پ» برای خلاصه‌نویسی جهت نوشتن عبارت «پروژه/پایان‌نامه/رساله»
\newcommand{\پ}{پروژه/پایان‌نامه/رساله }

%\newcommand\BackSlash{\char`\\}

%%%%%%%%%%%%%%%%%%%%%%%%%%
% \SepMark{-}

% تعریف و نحوه ظاهر شدن عنوان قضیه‌ها، تعریف‌ها، مثال‌ها و ...
\theoremstyle{definition}
\newtheorem{definition}{تعریف}[section]
\theoremstyle{theorem}
\newtheorem{theorem}[definition]{قضیه}
\newtheorem{lemma}[definition]{لم}
\newtheorem{proposition}[definition]{گزاره}
\newtheorem{corollary}[definition]{نتیجه}
\newtheorem{remark}[definition]{ملاحظه}
\theoremstyle{definition}
\newtheorem{example}[definition]{مثال}

%\renewcommand{\theequation}{\thechapter-\arabic{equation}}
%\def\bibname{مراجع}
\numberwithin{algorithm}{chapter}
\def\listalgorithmname{فهرست الگوریتم‌ها}
\def\listfigurename{فهرست تصاویر}
\def\listtablename{فهرست جداول}

% دستور های لازم برای تعریف ترجمهٔ دستورات الگوریتم
\makeatletter
\renewcommand{\algorithmicrequire}{\if@RTL\textbf{ورودی:}\else\textbf{Require:}\fi}
\renewcommand{\algorithmicensure}{\if@RTL\textbf{خروجی:}\else\textbf{Ensure:}\fi}
\renewcommand{\algorithmicend}{\if@RTL\textbf{پایان}\else\textbf{end}\fi}
\renewcommand{\algorithmicif}{\if@RTL\textbf{اگر}\else\textbf{if}\fi}
\renewcommand{\algorithmicthen}{\if@RTL\textbf{آنگاه}\else\textbf{then}\fi}
\renewcommand{\algorithmicelse}{\if@RTL\textbf{وگرنه}\else\textbf{else}\fi}
\renewcommand{\algorithmicfor}{\if@RTL\textbf{برای}\else\textbf{for}\fi}
\renewcommand{\algorithmicforall}{\if@RTL\textbf{برای هر}\else\textbf{for all}\fi}
\renewcommand{\algorithmicdo}{\if@RTL\textbf{انجام بده}\else\textbf{do}\fi}
\renewcommand{\algorithmicwhile}{\if@RTL\textbf{تا زمانی که}\else\textbf{while}\fi}
\renewcommand{\algorithmicloop}{\if@RTL\textbf{تکرار کن}\else\textbf{loop}\fi}
\renewcommand{\algorithmicrepeat}{\if@RTL\textbf{تکرار کن}\else\textbf{repeat}\fi}
\renewcommand{\algorithmicuntil}{\if@RTL\textbf{تا زمانی که}\else\textbf{until}\fi}
\renewcommand{\algorithmicprint}{\if@RTL\textbf{چاپ کن}\else\textbf{print}\fi}
\renewcommand{\algorithmicreturn}{\if@RTL\textbf{بازگردان}\else\textbf{return}\fi}
\renewcommand{\algorithmicand}{\if@RTL\textbf{و}\else\textbf{and}\fi}
\renewcommand{\algorithmicor}{\if@RTL\textbf{و یا}\else\textbf{or}\fi} % TODO add better translate
\renewcommand{\algorithmicxor}{\if@RTL\textbf{یا}\else\textbf{xor}\fi} % TODO add better translate
\renewcommand{\algorithmicnot}{\if@RTL\textbf{نقیض}\else\textbf{not}\fi}
\renewcommand{\algorithmicto}{\if@RTL\textbf{تا}\else\textbf{to}\fi}
\renewcommand{\algorithmicinputs}{\if@RTL\textbf{ورودی‌ها}\else\textbf{inputs}\fi}
\renewcommand{\algorithmicoutputs}{\if@RTL\textbf{خروجی‌ها}\else\textbf{outputs}\fi}
\renewcommand{\algorithmicglobals}{\if@RTL\textbf{متغیرهای عمومی}\else\textbf{globals}\fi}
\renewcommand{\algorithmicbody}{\if@RTL\textbf{انجام بده}\else\textbf{do}\fi}
\renewcommand{\algorithmictrue}{\if@RTL\textbf{درست}\else\textbf{true}\fi}
\renewcommand{\algorithmicfalse}{\if@RTL\textbf{نادرست}\else\textbf{false}\fi}
\renewcommand{\algorithmicendif}{\algorithmicend\textbf{ شرط }\algorithmicif}
\renewcommand{\algorithmicendfor}{\algorithmicend\textbf{ حلقهٔ }\algorithmicfor}
\renewcommand{\algorithmicendwhile}{\algorithmicend\textbf{ حلقهٔ }\algorithmicwhile}
\renewcommand{\algorithmicendloop}{\algorithmicend\textbf{ حلقهٔ }\algorithmicloop}
\renewcommand{\algorithmiccomment}[1]{\{{\itshape #1}\}}
\makeatletter

%%%%%%%%%%%%%%%%%%%%%%%%%%%%
%%% دستورهایی برای سفارشی کردن سربرگ صفحات:
%\newcommand{\SetHeader}[1]{
% دستور زیر معادل با گزینه twoside است.
%\csname@twosidetrue\endcsname
\pagestyle{fancy}
%% دستورات زیر سبک صفحات fancy را تغییر می‌دهد:
% O=Odd, E=Even, L=Left, R=Right
% در صورت oneside بودن، عنوان فصل، سمت چپ ظاهر می‌شود.
\fancyhead{}
\fancyhead[OL]{\small\leftmark}
\fancyhead[ER]{\small\leftmark}
\fancyhead[OR]{\footnotesize\rightmark}
\fancyhead[EL]{\footnotesize\rightmark}
\renewcommand{\headrulewidth}{0.75pt}
% شکل‌دهی شماره و عنوان فصل در سربرگ
\renewcommand{\chaptermark}[1]{\markboth{فصل~\thechapter:\ #1}{}}
\makeatletter
\renewcommand{\rightmark}[1]{\@title}
\makeatother
%}
%%%%%%%%%%%%%%%%%%%%%%%%%%%%
%\def\MATtextbaseline{1.5}
%\renewcommand{\baselinestretch}{\MATtextbaseline}
\doublespacing
%%%%%%%%%%%%%%%%%%%%%%%%%%%%%
% دستوراتی برای اضافه کردن کلمه «فصل» در فهرست مطالب

\newlength\mylenprt
\newlength\mylenchp
\newlength\mylenapp

\renewcommand\cftpartpresnum{\partname~}
\renewcommand\cftchappresnum{\chaptername~}
\renewcommand\cftchapaftersnum{:}

\settowidth\mylenprt{\cftpartfont\cftpartpresnum\cftpartaftersnum}
\settowidth\mylenchp{\cftchapfont\cftchappresnum\cftchapaftersnum}
\settowidth\mylenapp{\cftchapfont\appendixname~\cftchapaftersnum}
\addtolength\mylenprt{\cftpartnumwidth}
\addtolength\mylenchp{\cftchapnumwidth}
\addtolength\mylenapp{\cftchapnumwidth}

\setlength\cftpartnumwidth{\mylenprt}
\setlength\cftchapnumwidth{\mylenchp}	

\makeatletter
{\def\thebibliography#1{\chapter*{\refname\@mkboth
   {\uppercase{\refname}}{\uppercase{\refname}}}\list
   {[\arabic{enumi}]}{\settowidth\labelwidth{[#1]}
   \rightmargin\labelwidth
   \advance\rightmargin\labelsep
   \advance\rightmargin\bibindent
   \itemindent -\bibindent

   \listparindent \itemindent
   \parsep \z@
   \usecounter{enumi}}
   \def\newblock{}
   \sloppy
   \sfcode`\.=1000\relax}}
   
%اگر مایلید در شماره گذاری حرفی و ابجد به جای آ از الف استفاده شود دستورات زیر را فعال کنید.   
%\def\@Abjad#1{%
%  \ifcase#1\or الف\or ب\or ج\or د%
%           \or هـ\or و\or ز\or ح\or ط%
%           \or ی\or ک\or ل\or م\or ن%
%           \or س\or ع\or ف\or ص%
%           \or ق\or ر\or ش\or ت\or ث%
%            \or خ\or ذ\or ض\or ظ\or غ%
%            \else\@ctrerr\fi}
%
% \def\abj@num@i#1{%
%   \ifcase#1\or الف\or ب\or ج\or د%
%            \or هـ‍\or و\or ز\or ح\or ط\fi

%   \ifnum#1=\z@\abjad@zero\fi}   
%  
%   \def\@harfi#1{\ifcase#1\or الف\or ب\or پ\or ت\or ث\or

% ج\or چ\or ح\or خ\or د\or ذ\or ر\or ز\or ژ\or س\or ش\or ص\or ض\or ط\or ظ\or ع\or غ\or

% ف\or ق\or ک\or گ\or ل\or م\or ن\or و\or ه\or ی\else\@ctrerr\fi}

%
\makeatother

%%% امکان درج کد در سند
% در این قسمت رنگ، قلم و قالب‌بندی قسمت‌های مختلف یک کد تعیین می‌شود. 
\lstdefinestyle{myStyle}{
	basicstyle=\ttfamily, % whole listing /w verbatim font
	keywordstyle=\color{blue}\bfseries, % bold black keywords
	identifierstyle=, % nothing happens
	commentstyle=\color{LimeGreen}, % green comments
	stringstyle=\ttfamily\color{red}, % red typewriter font for strings
	showstringspaces=false % no special string spaces
	breaklines=true,
	breakatwhitespace=false,
	numbers=right, % line number formats
	numberstyle=\footnotesize\lr,
	numbersep=-10pt,
	frame=single,
	captionpos=b,
	captiondirection=RTL
}
\lstset{style=myStyle} % command to set default style
\def\lstlistingname{\rl{برنامهٔ}}
\def\lstlistlistingname{\rl{فهرست برنامه‌ها}}


% for numbering subsubsections
\setcounter{secnumdepth}{3}
%to include subsubsections in the table of contents
\setcounter{tocdepth}{3}

\makeatletter
\renewcommand{\p@subfigure}{\thefigure.}
\makeatother


% مشخصات پایان‌نامه را در فایلهای faTitle و enTitle وارد نمایید.
% !TeX root=../main.tex
% در این فایل، عنوان پایان‌نامه، مشخصات خود، متن تقدیمی‌، ستایش، سپاس‌گزاری و چکیده پایان‌نامه را به فارسی، وارد کنید.
% توجه داشته باشید که جدول حاوی مشخصات پروژه/پایان‌نامه/رساله و همچنین، مشخصات داخل آن، به طور خودکار، درج می‌شود.
%%%%%%%%%%%%%%%%%%%%%%%%%%%%%%%%%%%%
% دانشگاه خود را وارد کنید
\university{دانشگاه تهران}
% پردیس دانشگاهی خود را اگر نیاز است وارد کنید (مثال: فنی، علوم پایه، علوم انسانی و ...)
\college{پردیس دانشکده‌های فنی}
% دانشکده، آموزشکده و یا پژوهشکده  خود را وارد کنید
\faculty{دانشکده مهندسی برق و کامپیوتر}
% گروه آموزشی خود را وارد کنید (در صورت نیاز)
\department{گروه هوش مصنوعی و رباتیکز}
% رشته تحصیلی خود را وارد کنید
\subject{مهندسی کامپیوتر}
% گرایش خود را وارد کنید
\field{هوش مصنوعی و رباتیکز}
% عنوان پایان‌نامه را وارد کنید
%\title{نوشتن پروژه، پایان‌نامه و رساله با استفاده از کلاس 
%\lr{tehran-thesis}}
\title{سیستم هشدار زودهنگام زلزله از طریق تلفن همراه}

% نام استاد(ان) راهنما را وارد کنید
\firstsupervisor{دکتر هادی مرادی}
\firstsupervisorrank{دانشیار}
\secondsupervisor{دکتر محمود رضا هاشمی}
\secondsupervisorrank{دانشیار}
% نام استاد(دان) مشاور را وارد کنید. چنانچه استاد مشاور ندارید، دستورات پایین را غیرفعال کنید.
\firstadvisor{دکتر علی مرادی}
\firstadvisorrank{دانشیار}
%\secondadvisor{دکتر مشاور دوم}
% نام داوران داخلی و خارجی خود را وارد نمایید.
\internaljudge{دکتر داور داخلی}
\internaljudgerank{دانشیار}
\externaljudge{دکتر داور خارجی}
\externaljudgerank{دانشیار}
\externaljudgeuniversity{دانشگاه داور خارجی}
% نام نماینده کمیته تحصیلات تکمیلی در دانشکده \ گروه
\graduatedeputy{دکتر نماینده}
\graduatedeputyrank{دانشیار}
% نام دانشجو را وارد کنید
\name{رضا}
% نام خانوادگی دانشجو را وارد کنید
\surname{کرباسی}
% شماره دانشجویی دانشجو را وارد کنید
\studentID{۸۱۰۱۹۸۲۳۳}
% تاریخ پایان‌نامه را وارد کنید
\thesisdate{بهمن ۱۴۰۰}
% به صورت پیش‌فرض برای پایان‌نامه‌های کارشناسی تا دکترا به ترتیب از عبارات «پروژه»، «پایان‌نامه» و «رساله» استفاده می‌شود؛ اگر  نمی‌پسندید هر عنوانی را که مایلید در دستور زیر قرار داده و آنرا از حالت توضیح خارج کنید.
%\projectLabel{پایان‌نامه}

% به صورت پیش‌فرض برای عناوین مقاطع تحصیلی کارشناسی تا دکترا به ترتیب از عبارت «کارشناسی»، «کارشناسی ارشد» و «دکتری» استفاده می‌شود؛ اگر نمی‌پسندید هر عنوانی را که مایلید در دستور زیر قرار داده و آنرا از حالت توضیح خارج کنید.
%\degree{کارشناسی ارشد}
%%%%%%%%%%%%%%%%%%%%%%%%%%%%%%%%%%%%%%%%%%%%%%%%%%%%
%% پایان‌نامه خود را تقدیم کنید! %%
\dedication
{
{\Large تقدیم به:}\\
\begin{flushleft}{
	\huge
%	همسر و فرزندانم\\
%	\vspace{7mm}
%	و\\
	\vspace{7mm}
	پدر و مادرم
}
\end{flushleft}
}
%% متن قدردانی %%
%% ترجیحا با توجه به ذوق و سلیقه خود متن قدردانی را تغییر دهید.
\acknowledgement{
سپاس خداوندگار حکیم را که با لطف بی‌کران خود، آدمی را به زیور عقل آراست.

در آغاز وظیفه‌  خود  می‌دانم از زحمات بی‌دریغ اساتید خود،  جناب آقای دکتر هادی مرادی و دکتر علی مرادی ، صمیمانه تشکر و  قدردانی کنم که در طول انجام این پایان‌نامه با نهایت صبوری همواره راهنما و مشوق من بودند و قطعاً بدون راهنمایی‌های ارزنده‌ ایشان، این مجموعه به انجام نمی‌رسید.
%
%از جناب آقای دکتر ... که  زحمت مشاوره‌، بازبینی و تصحیح این پایان‌نامه را تقبل فرمودند کمال امتنان را دارم.

%از همکاری و مساعدت‌های دکتر ... مسئول تحصیلات تکمیلی و سایر کارکنان دانشکده بویژه سرکار خانم ... کمال تشکر را دارم.

همچنین از دوستان آزمایشگاه سیستم سیستم‌های هوشمند سپاس‌گزارم که با همفکری مرا صمیمانه و مشفقانه یاری داده‌اند.
%
%با سپاس بی‌دریغ خدمت دوستان گران‌مایه‌ام، خانم‌ها ... و آقایان ... در آزمایشگاه ...، که با همفکری مرا صمیمانه و مشفقانه یاری داده‌اند.
%
%و در پایان، بوسه می‌زنم بر دستان خداوندگاران مهر و مهربانی، پدر و مادر عزیزم و بعد از خدا، ستایش می‌کنم وجود مقدس‌شان را و تشکر می‌کنم از خانواده عزیزم به پاس عاطفه سرشار و گرمای امیدبخش وجودشان، که بهترین پشتیبان من بودند.
}
%%%%%%%%%%%%%%%%%%%%%%%%%%%%%%%%%%%%
%چکیده پایان‌نامه را وارد کنید
\fa-abstract{
چکیده
\todo{چکیده باید بیاید}
%این راهنما، نمونه‌ای از قالبِ پروژه، پایان‌نامه و رسالهٔ دانشگاه تهران می‌باشد که با استفاده از کلاس 
%\lr{tehran-thesis}
%و بستهٔ زی‌پرشین در \lr{\LaTeX}{} تهیه شده است. این قالب به گونه‌ای طراحی شده است که مطابق با دستورالعمل نگارش و تدوین پایان‌نامه کارشناسی ارشد و دکتری، مورخ ۹۳/۰۶/۰۳ پردیس دانشکده‌های فنی دانشگاه تهران باشد و حروف‌چینی بسیاری از قسمت‌های آن، مطابق با استاندارد قالب‌های فارسی پایان‌نامه در لاتک، به طور خودکار انجام می‌شود.
%
%چکیده بخشی از پایان‌نامه است که خواننده را به مطالعه آن علاقمند می‌کند و یا از آن می‌گریزاند. چکیده باید ترجیحاً‌ در یک صفحه باشد. در نگارش چکیده نکات زیر باید رعایت شود. متن چکیده باید مزین به کلمه‌ها و عبارات سلیس، آشنا، بامعنی و روشن باشد. بگونه‌ای که با حدود ۳۰۰ تا ۵۰۰ کلمه بتواند خواننده را به خواندن پایان‌نامه راغب نماید. چکیده، جدای از پایان‌نامه باید به تنهایی گویا و مستقل باشد. در چکیده باید از ذکر منابع، اشاره به جداول و نمودارها اجتناب شود.تمیز بودن مطلب، نداشتن غلط‌های املایی یا دستور زبانی و رعایت دقت و تسلسل روند نگارش چکیده از نکات مهم دیگری است که باید درنظر گرفته شود. در چکیده پایان‌نامه باید از درج مشخصات مربوط به پایان‌نامه خودداری شود.
%چکیده باید منعکس‌کننده اصل موضوع باشد. در چکیده باید اهداف تحقیق مورد توجه قرار گیرد. تأکید روی اطلاعات تازه (یافته‌ها) و اصطلاحات جدید یا نظریه‌ها، فرضیه‌ها، نتایج و پیشنهادها متمرکز شود. اگر در پایان‌نامه روش نوینی برای اولین بار ارائه می‌شود و تا به حال معمول نبوده است، با جزئیات بیشتری ذکر شود. شایان ذکر است چکیده فارسی و انگلیسی باید حتماً به تأیید استاد راهنما رسیده باشد.
%
%کلمات کلیدی در انتهای چکیده فارسی و انگلیسی آورده می‌شود. محتوای چکیده‌ها بر اساس موضوع و گرایش تحقیق طبقه‌بندی می‌شود و به همین جهت وجود کلمات شاخص و کلیدی، مراکز اطلاعاتی  را در طبقه‌بندی دقیق و سریع پایان‌نامه یاری می‌دهد. کلمات کلیدی، راهنمای نکات مهم موجود در پایان‌نامه هستند. بنابراین باید در حد امکان کلمه‌ها یا عباراتی انتخاب شود که ماهیت، محتوا و گرایش کار را به وضوح روشن نماید.
}
% کلمات کلیدی پایان‌نامه را وارد کنید
%\keywords{حداکثر ۵ کلمه یا عبارت، متناسب با عنوان، قالب پایان‌نامه، لاتک}
\keywords{هشدار زود هنگام زلزله , \todo{ادامه واژگان کلیدی}}
% انتهای وارد کردن فیلد‌ها
%%%%%%%%%%%%%%%%%%%%%%%%%%%%%%%%%%%%%%%%%%%%%%%%%%%%%%

% مشخصات انگلیسی پایان‌نامه
% !TeX root=../main.tex
% در این فایل، عنوان پایان‌نامه، مشخصات خود و چکیده پایان‌نامه را به انگلیسی، وارد کنید.

%%%%%%%%%%%%%%%%%%%%%%%%%%%%%%%%%%%%
\latinuniversity{University of Tehran}
\latincollege{College of Engineering}
\latinfaculty{Faculty of Electrical and Computer Engineering}
\latindepartment{Machine Intelligence and Robotics}
\latinsubject{Computer Engineering}
\latinfield{Algorithms and Computation}
\latintitle{An earthquake early warning system using smartphones}
\firstlatinsupervisor{Dr. Hadi Moradi}
\secondlatinsupervisor{Dr. Mahmoud Reza Hashemi}
\firstlatinadvisor{Dr. Ali Moradi}
%\secondlatinadvisor{Second Advisor}
\latinname{Reza}
\latinsurname{Karbasi}
\latinthesisdate{Jan 2022}
\latinkeywords{\todo{english keywords}}
\en-abstract{
\todo{english abstract}
%This thesis studies on writing projects, theses and dissertations using tehran-thesis class. It ...
}


% تنظیمات و تعاریف واژه‌نامه و اختصارات
\input{./tex/glossaries-settings}
\input{./tex/words}
\input{./tex/acronyms}

\begin{document}

\pagenumbering{adadi} % یک، دو، ...
\input{./tex/thesis_preamble}

\pagestyle{fancy}
\pagenumbering{arabic} % 1, 2, ...

%% !TeX root=../main.tex
\chapter{مقدمه}
%\thispagestyle{empty} 
\section{مقدمه} 		% فصل اول: مقدمه
%% !TeX root=../main.tex
\chapter{مروری بر مطالعات انجام شده}
%\thispagestyle{empty} 
\section{مقدمه} 		% فصول دوم: مروری بر مطالعات انجام شده
%\chapter{روش تحقیق}

	\section{مقدمه}
	هدف اصلی این پایان‌نامه تشخیص زود هنگام زلزله است. برای تشخیص زودهنگان زلزله ابتدا نیاز به تخمینی از مکان وقوع لرزش است و پس از آن باید با کمک روابط تضعیف تخمینی از میزان لرزش در مکان های هدف ارائه شود و در انتها بر اساس آن تخمین هشدارهای مربوط صادر شوند.
\\ 
پس ابتدا روش‌های پیشنهاد شده برای تخمین مکان لرزش توضیح داده شده‌است.
سپس بستر های پیاده‌سازی این الگوریتم‌ها معرفی شد‌‌ه‌است.
قدم بعدی بررسی روابط تضعیف و نحوه تخمین آن‌ها است. در انتهای فصل به بررسی نحوه تخمین میزان لرزش در مکان هدف پرداخته شده‌است.

	\section{بسترهای پیاده سازی}
	
به منظور بررسی کارآیی روش‌های به کار گرفته شده نیاز است تا آن ها را در محیط های مختلفی آزمایش کنیم. اصلی ترین بستری که نیاز است کارآیی مناسبی در آن وجود داشته باشد، محیط واقعی است یعنی الگوریتم‌های مورد بررسی بر گوشی‌های تلفن همراه و محیط سرور پیاده سازی شوند و تشخیص‌ها به خوبی انجام شده ، هشدارها صادر شوند. اما از آنجایی که هزینه‌های مالی و معنوی این کار برای بررسی اولیه الگوریتم زیاد است، نیاز است تا این روش‌ها بر روی دادگان شبیه‌سازی شده و دادگان واقعی گذشته اعمال شود.


\subsection{پیاده‌سازی عملی سیستم}


\subsection{ استفاده از شبیه‌ساز }
همان‌طور که به میان آمد، برای بهینه کردن پارامترهای الگوریتم ها نیاز است الگوریتم‌ها را محیط هایی ساده پیاده‌سازی کرده و کارآیی آن‌ها را بیشینه کنیم. به همین منظور برای این پژوهش شبیه‌سازی طراحی شده‌است.


	
	\section{تهیه دادگان مورد نیاز}

	\section{الگوریتم های تعین مکان}

	\section{تعین پارامتر های تضعیف}

	\section{نحوه هشدار به مکان ثالث}

	\section{جمع بندی}
		% فصل سوم: روش تحقیق
%\chapter{نتایج}
%\thispagestyle{empty} 
\section{مقدمه} 		% فصل چهارم: نتایج
%\chapter{بحث و نتیجه گیری}
%\thispagestyle{empty} 
\section{مقدمه} 		% فصل پنجم: بحث و نتیجه‌گیری

% مراجع
% اگر از استیل‌های natbib استفاده می‌کنید باید دو خط را در فایل commands.tex تغییر دهید.
\pagestyle{empty}
{
\small
\onehalfspacing
\bibliographystyle{plain-fa} % or plainnat-fa for author-date
\bibliography{./tex/MyReferences}
}

\pagestyle{fancy}

% \appendix
% فصلهای پس از این قسمت به عنوان ضمیمه خواهند آمد.

% دستورات لازم برای تبدیل «فصل آ» به «پیوست آ» در فهرست مطالب
\addtocontents{toc}{
    \protect\renewcommand\protect\cftchappresnum{\appendixname~}%
    \protect\setlength{\cftchapnumwidth}{\mylenapp}}
    
% دستورات لازم برای شماره‌گذاری صفحات پیوست‌ها بشکل آ-۱ (فعلا با glossaries سازگار نیست)
% \let\Chapter\chapter
%\pretocmd{\chapter}{
%  \clearpage
%  \pagenumbering{arabic}
%  \renewcommand*{\thepage}{\rl{\thechapter-\arabic{page}}}}{}{}
%%%%%%%%%%%%%%%%%%%%%%%%%%%%%%%%%%%%%
        

\include{./tex/appendix1}		% پیوست اول: آشنایی مقدماتی با لاتک
%\include{./tex/appendix2}		% پیوست دوم: جدول، نمودار و الگوریتم در لاتک
% !TeX root=../main.tex
\chapter{مراجع، واژه‌نامه و حاشیه‌نویسی}
\label{app:refMan}
%\thispagestyle{empty}

\section{مراجع و نقل‌قول‌ها}
\label{sec:refUsage}
منابعِ پایان‌نامه، پایه و اساس تحقیق شما به حساب می‌آیند و ضرورت انجام مطالعه و روش‌های به کار رفته در بسیاری از قسمت‌های آن، به کمک منابع صورت می‌گیرد. در استفاده از مراجع علمی در پایان‌نامه، باید سعی کنید بیشتر از
\textbf{منابع چاپ‌شده و مهم}
استفاده کنید و
\emph{ارجاع به داده‌های چاپ نشده، خلاصه‌ها و پایان‌نامه‌ها، سبب به‌هم‌خوردگی و کاهش اعتبار قسمت ارجاع منابع می‌شود.}
استفاده از منابع و نقل قول‌هایی به تحقیق شما ارزش می‌دهند که
\textbf{در راستای هدف تحقیق بوده و به آن اعتبار ببخشند.}
برخی از دانش‌جویان تصوّر می‌کنند که کثرت نقل‌قول‌ها و ارجاعات زیاد، مهم‌ترین معیار علمی شدن پایان‌نامه است؛ حال آنکه استناد به تعداد کثیری از منابع بدون مطالعه دقیق آنها و استفادهٔ مستقیم در پایان‌نامه، می‌تواند نشان‌دهندهٔ عدم احساس امنیت نویسنده باشد!

دو روش برای استفاده از نتایج، جملات، داده‌ها و روش‌های دیگران وجود دارد. یکی نقل‌قول مستقیم و دقیق است و دیگری استفاده غیرمستقیم در متن مقاله، که در ادامه به قواعد این دو نوع نقل‌قول و ارجاع‌دهی اشاره می‌کنیم:
\begin{description}
	\item[نقل‌قول مستقیم:]
	نقل‌قول مستقیم باید دقیق و بدون هیچ تغییری در جملات باشد. بهتر است این‌گونه نقل‌قول‌ها تا حد امکان کوتاه باشد. جملات کوتاه داخل گیومه قرار می‌گیرند و باید به منبع دقیق آن، طبق روش ارجاع‌دهی به منابع، اشاره شود. به عنوان مثال در
	\cite{persianbib87userguide}
	آمده است که:
	\begin{quote}
		«با استفاده از فیلد
		\lr{AUTHORFA}
		می‌توان معادل فارسی نام نویسندگان مقالات لاتین را در متن داشت. معمولاً در اسناد فارسی خواسته می‌شود که پس از ذکر معادل فارسی نام نویسنده، نام لاتین نویسنده(ها) به عنوان پاورقی درج شود
		\citep{persianbib87userguide}.»
	\end{quote}
	\item[نقل‌قول غیرمستقیم:]
	نقل‌قول غیرمستقیم به معنی استفاده از ایده‌ها، نتایج، روش‌ها و داده‌های دیگران در درون متنِ پایان‌نامه، ولی به سبک خودتان و متناسب و هماهنگ با روند پایان‌نامهٔ شماست. در این حالت نیز باید متناسب با شیوهٔ ارجاع‌دهی به آن استناد شود.
\end{description}

با توجه به وجود سبک‌های مختلف ارجاع‌دهی، باید
\textbf{روش قابل قبول و یکسانی}
در طول پایان‌نامه برای اشاره به مراجع در متن و همچنین تهیه فهرست مراجع در انتهای پایان‌نامه بکار رود. مثلاً برای پایان‌نامه‌های مهندسی می‌توان از سبک ارجاع‌دهی
\lr{IEEE}%
\LTRfootnote{\url{http://www.ieee.org/documents/ieeecitationref.pdf}}
یا
\lr{acm}
استفاده کرد. طبیعتاً باید تناظر یک‌به‌یک بین فهرست مراجع در انتهای گزارش و مراجع مورد استفاده در متن باشد%
\footnote{البته گاهی ممکن است محقق مرجعی را مورد مطالعه قرار داده لیکن در متن به آن اشاره نکرده باشد؛ برخی معتقدند در این موارد نیز آوردن آن در فهرست مراجع، اشکالی ندارد، به این شرط که از عنوان «فهرست منابع» به جای «فهرست مراجع» استفاده شود.}.

برای سهولت مدیریت مراجعِ \پ%
، اکیداً توصیه می‌شود از یک ابزار «مدیریت منابع» (با خروجی
\texorpdfstring{\lr{Bib\TeX}}{Bib\TeX}%
) همچون
\lr{Mendeley}،
\lr{Zotero},
\lr{EndNote}
یا
\lr{Citavi}
استفاده کنید.

\subsection{ مدیریت مراجع با  \texorpdfstring{\lr{Bib\TeX}}{Bib\TeX}}
در بخش \ref{Sec:Ref} اشاره شد که با دستور 
 \lr{\textbackslash bibitem}
  می‌توان یک مرجع را تعریف نمود و با فرمان
 \lr{\textbackslash cite}
  به آن ارجاع داد. این روش برای تعداد مراجع زیاد و تغییرات آنها مناسب نیست. برای مدیریت منابع زیاد، سه بستهٔ
\lr{BibTeX} (پیش‌فرض),
\lr{natbib}
(ارجاع‌دهی در متن به صورت نویسنده-سال)
و \lr{BibLaTeX} (جدید و منعطف‌پذیر)
وجود دارند. در ادامه توضیحاتی در مورد مدیریت منابع با \lr{BibTeX} و \lr{natbib} در زی‌پرشین خواهیم آورد که همراه با توزیع‌های معروف تِک عرضه می‌شوند
\footnote{روش \lr{BibLaTeX} هنوز برای متون فارسی به درستی ترجمه نشده است.}.

یکی از روش‌های قدرتمند و انعطاف‌پذیر برای نوشتن مراجعِ مقالات و مدیریت مراجع در لاتک، استفاده از  \lr{BibTeX} است.
 روش کار با بیب‌تک به این صورت است که مجموعهٔ همهٔ مراجعی را که در \پ استفاده کرده یا خواهیم کرد، 
در پروندهٔ جداگانه‌ای با پسوند
\lr{bib}
نوشته و به آن فایل در سند خودمان به صورت مناسب لینک می‌دهیم.
 کنفرانس‌ها یا مجله‌های گوناگون برای نوشتن مراجع، قالب‌ها یا قراردادهای متفاوتی دارند که به آنها استیل‌های مراجع گفته می‌شود.
 در این حالت به کمک ‌استیل‌های بیب‌تک خواهید توانست تنها با تغییر یک پارامتر در پروندهٔ ورودی خود، مراجع را مطابق قالب موردنظر تنظیم کنید. 
 بیشتر مجلات و کنفرانس‌های معتبر یک فایل سبک
 (\lr{BibTeX Style})
با پسوند \lr{bst} در وب‌گاه خود می‌گذارند که برای همین منظور طراحی شده است.

به جز نوشتن مقالات، این سبک‌ها کمک بسیار خوبی برای تهیهٔ مستندات علمی همچون پایان‌نامه‌هاست که فرد می‌تواند هر قسمت از کارش را که نوشت مراجع مربوطه را به بانک مراجع خود اضافه نماید. با داشتن چنین بانکی از مراجع، وی خواهد توانست به راحتی یک یا چند ارجاع به مراجع و یا یک یا چند بخش را حذف یا اضافه ‌نماید؛ 
مراجع به صورت خودکار مرتب شده و
\textbf{فقط مراجع ارجاع داده شده در قسمت کتاب‌نامه خواهندآمد.}
قالب مراجع به صورت یکدست مطابق سبک داده شده بوده و نیازی نیست که کاربر درگیر قالب‌دهی به مراجع باشد. 

\subsection{سبک‌های مورد تأیید دانشگاه تهران}
طبق «دستورالعمل نگارش و تدوین پایان‌نامه» دانشگاه تهران در
\cite{UTThesisGuide}،
ارجاع در متن می‌تواند مطابق با هر یک از دو الگوی هاروارد یا ونکوور باشد:
\singlespacing
\begin{description}
	\item[سیستم نویسنده-سال (هاروارد):]
	ذکر نام نویسنده و سال نشر در متن. در این الگو مراجع بر اساس حروف الفبا تنظیم می‌گردند.
	\item[سیستم شماره‌دار (ونکوور):]
	ارجاع به مراجع به کمک شماره در متن. در این الگو شماره هر مرجع به ترتیب ظاهر شدن آن در متن در داخل کروشه قرار می‌گیرد. فهرست مراجع نیز بر اساس شماره مرجع (نه حروف الفبا) تنظیم می‌گردد.
\end{description}
\doublespacing

در مدیریت منابع با
\lr{\textbf{BibTeX}}،
ارجاع‌ها در متن تنها به شکل
\textbf{شماره‌دار (ونکوور)}
امکان‌پذیر است، گرچه فهرست مراجع می‌تواند با روش‌های مختلف مرتب شود. اگر بخواهیم ارجاع‌ها در متن به صورت
\textbf{نویسنده-سال (هاروارد)}
باشد باید از بستهٔ
\lr{\textbf{natbib}}\LTRfootnote{Natural Sciences Citations \& References}
و استیل‌های مختلف آن استفاده کنیم.

هنگام استفاده از روش نویسنده-سال نوع پرانتزگذاری‌ها در وسط و انتهای جمله با هم فرق خواهد داشت. به مثال زیر مطابق با دستورالعمل
\cite{UTThesisGuide}
توجه کنید:

\textit{
ابتدا
\cite{Khalighi87xepersian}
بستهٔ زی‌پرشین را برای حروف‌چینی فارسی اختراع کرد. بعدها سبک‌های ارجاع‌دهی فارسی و قالب‌های پایان‌نامه نیز مبتنی بر آن ساخته شد
\citep{persianbib87userguide}.
ارجاع‌دهی به مراجع لاتین نیز در زی‌پرشین امکان‌پذیر است. مثلاً
\citelatin{Gonzalez02book}
یک کتاب انگلیسی است و به راحتی به مقالات انگلیسی نیز می‌توان ارجاع داد
\citeplatin{kim2016integrated}.}

در این مثال، ۴ ارجاع در وسط و انتهای جمله به مراجع فارسی و انگلیسی آمده است. وقتی از سیستم نویسنده-سال استفاده می‌کنید، بهتر است ارجاع‌های آخر جمله کلاً داخل پرانتر بیاید؛ بدین منظور باید به جای
\verb|\cite|
از
\verb|\citep|
استفاده کنید. اما در سیستم شماره‌دار چون تمام ارجاع‌ها داخل کروشه می‌آیند این امر اهمیت ندارد.\\
نمی‌توانید در متن فارسی، اسم لاتین محقق خارجی را بیاورید و برای جلوگیری از ایجاد ابهام، صرف‌نظر از نام لاتین هم مجاز نیست! توصیه می‌شود که نام محقق خارجی در متن با حروف فارسی و در پاورقی اسم تمام نویسندگان به صورت انگلیسی آورده شود. نحوهٔ رعایت این نکته را می‌توانید در کد مثال بالا ببینید.

گرچه در تمپلت ورد
\cite{UTThesisGuide}،
به صراحت ذکر شده که بهتر است برای پایان‌نامه‌های مهندسی از سبک 
\lr{IEEE}
استفاده شود (که از سیستم ونکوور تبعیت می‌کند)، اما ترتیب فهرست مراجع در
\lr{IEEE}
بر اساس ترتیب ارجاع در متن بوده و
\emph{مراجع انگلیسی و فارسی از هم تفکیک نمی‌شوند}
که متضاد با دستورالعمل
\cite{UTThesisGuide}
و نیز متضاد عرف اکثر پایان‌نامه‌های فارسی است.
بنابراین دقیقاً نمی‌توان سبک خاصی را برای مراجع پایان‌نامه‌های دانشگاه تهران اجبار کرد. مهم این است که
\textbf{سبک ارجاع‌دهی در تمام طول یک کتابچه}
(مثلاً پایان‌نامه، مقالات یک مجله یا کل یک کتاب) یکسان باشد. بهتر است
\textbf{بسته به حوزه پایان‌نامه}،
در این مورد با استاد راهنمای خود مشورت کنید.

\subsection{سبک‌های فارسی قابل استفاده در زی‌پرشین}
تعدادی از سبک‌های فارسی بسته
\lr{Persian-bib}%
\footnote{ برای اطلاع بیشتر به راهنمای بستهٔ
\lr{Persian-bib}
مراجعه فرمایید.}
که برای  زی‌پرشین آماده شده‌اند، عبارتند از:

\singlespacing
\begin{itemize}
\item \emph{سبک‌های شماره‌دار}:
	\begin{description}
	\item [unsrt-fa.bst] این سبک متناظر با \lr{unsrt.bst} می‌باشد. مراجع به ترتیب ارجاع در متن ظاهر می‌شوند.
	\item [plain-fa.bst] این سبک متناظر با \lr{plain.bst} می‌باشد. مراجع بر اساس نام‌خانوادگی نویسندگان، به ترتیب صعودی مرتب می‌شوند.
	 همچنین ابتدا مراجع فارسی و سپس مراجع انگلیسی خواهند آمد.
	\item [acm-fa.bst] این سبک متناظر با \lr{acm.bst} می‌باشد. شبیه \lr{plain-fa.bst} است.  قالب مراجع کمی متفاوت است. اسامی نویسندگان انگلیسی با حروف بزرگ انگلیسی نمایش داده می‌شوند. (مراجع مرتب می‌شوند)
	\item [ieeetr-fa.bst] این سبک متناظر با \lr{ieeetr.bst} می‌باشد. (مراجع مرتب نمی‌شوند)
	\end{description}
	
\item \emph{سبک‌های نویسنده-سال}:
	\begin{description}
	\item [plainnat-fa.bst] این سبک متناظر با \lr{plainnat.bst} می‌باشد. نیاز به بستهٔ \lr{natbib} دارد. (مراجع مرتب می‌شوند)
	\item [chicago-fa.bst] این سبک متناظر با \lr{chicago.bst} می‌باشد. نیاز به بستهٔ \lr{natbib} دارد. (مراجع مرتب می‌شوند)
	\item [asa-fa.bst] این سبک متناظر با \lr{asa.bst} می‌باشد. نیاز به بستهٔ \lr{natbib} دارد. (مراجع مرتب می‌شوند)
	\end{description}
\end{itemize}
\doublespacing

با استفاده از استیل‌های فوق می‌توانید به انواع مختلفی از مراجع فارسی و لاتین ارجاع دهید.
به عنوان مثال‌هایی از
\textbf{مراجع انگلیسی}،
مرجع
\cite{Baker02limits}\footnote{چون فیلد \lr{authorfa} برای این مرجع تعریف نشده در سبک نویسنده-سال با حروف لاتین به آن در متن ارجاع می‌شود که غلط است.}
مقالهٔ یک ژورنال، مرجع
\cite{Amintoosi09video}
مقالهٔ یک کنفرانس، مرجع
\citelatin{Gonzalez02book}
یک کتاب، مرجع
\cite{Khalighi07MscThesis}
پایان‌نامهٔ کارشناسی ارشد و مرجع
\citelatin{Borman04thesis}
یک رسالهٔ دکتری می‌باشد.\\
همچنین در میان
\textbf{مراجع فارسی},
مرجع
\cite{Vahedi87}
مقالهٔ یک مجله، مرجع
\cite{Amintoosi87afzayesh}
مقالهٔ یک کنفرانس، مرجع
\cite{Pedram80osool}
یک کتاب ترجمه‌شده با ذکر مترجمان و ویراستاران، مرجع
\cite{Pourmousa88mscThesis}
پایان‌نامهٔ کارشناسی ارشد%
\footnote{همان‌طور که در بخش
\ref{sec:refUsage}
اشاره شد، بهتر است زیاد از پایان‌نامه‌ها در مراجع استفاده نکنید.}،
مرجع
\cite{Omidali82phdThesis}
یک رسالهٔ دکتری و مراجع
\cite{persianbib87userguide, Khalighi87xepersian}
نمونه‌های متفرقه هستند.

\subsection{ساختار فایل مراجع}
برای استفاده از بیب‌تک باید مراجع خود را در یک فایل با پسوند \lr{bib} ذخیره نمایید. یک فایل \lr{bib} در واقع یک پایگاه داده از مراجع%
\LTRfootnote{Bibliography Database}
شماست که هر مرجع در آن به عنوان یک رکورد از این پایگاه داده
با قالبی خاص ذخیره می‌شود. به هر رکورد یک مدخل%
\LTRfootnote{Entry}
گفته می‌شود. یک نمونه مدخل برای معرفی کتاب \lr{Digital Image Processing} در ادامه آمده است:

\singlespacing
\begin{LTR}
\begin{verbatim}
@BOOK{Gonzalez02image,
  AUTHOR     = {Gonzalez,, Rafael C. and Woods,, Richard E.},
  TITLE      = {Digital Image Processing},
  PUBLISHER  = {Prentice-Hall, Inc.},
  YEAR       = {2006},
  ISBN       = {013168728X},
  EDITION    = {3rd},
  ADDRESS    = {Upper Saddle River, NJ, USA}
}
\end{verbatim}
\end{LTR}
\doublespacing

در مثال فوق، \lr{@BOOK} مشخصهٔ شروع یک مدخل مربوط به یک کتاب و \lr{Gonzalez02book} برچسبی است که به این مرجع منتسب شده است.
 این برچسب بایستی یکتا باشد. برای آنکه بتوان
\textbf{برچسب مراجع}
 را به راحتی به خاطر سپرد و حتی‌الامکان برچسب‌ها متفاوت با هم باشند، معمولاً از قوانین خاصی به این منظور استفاده می‌شود. یک قانون می‌تواند
\textbf{فامیل نویسنده اول + دورقم سال نشر + اولین کلمهٔ عنوان اثر}
باشد. به
\lr{AUTHOR}، \lr{TITLE}، $\dots$ و \lr{ADDRESS}
فیلدهای این مدخل گفته می‌شود، که هر یک با مقادیر مربوط به مرجع پر شده‌اند. ترتیب فیلدها مهم نیست. 

انواع متنوعی از مدخل‌ها برای اقسام مختلف مراجع همچون کتاب، مقالهٔ کنفرانس و مقالهٔ ژورنال وجود دارد که برخی فیلدهای آنها با هم متفاوت است. 
نام فیلدها بیانگر نوع اطلاعات آن می‌باشد. مثالهای ذکر شده در فایل \lr{MyReferences.bib} کمک خوبی برای شما خواهد بود. 
%این فایل یک فایل متنی بوده و با ویرایشگرهای معمول همچون \lr{Notepad++} قابل ویرایش می‌باشد. برنامه‌هایی همچون 
%\lr{TeXMaker}
% امکاناتی برای نوشتن این مدخل‌ها دارند و به صورت خودکار فیلدهای مربوطه را در فایل \lr{bib}  شما قرار می‌دهند.  
با استفاده از سبک‌های فارسی آماده شده، محتویات هر فیلد می‌تواند به فارسی نوشته شود؛ ترتیب مراجع و نحوهٔ چینش فیلدهای هر مرجع را سبک مورد استفاده  مشخص خواهد کرد.

\textbf{در فایل 
\lr{MyReferences.bib}
 که همراه با این \پ هست، مثال‌های مختلفی از مراجع آمده‌اند که برای درج مراجع خود، تنها کافیست مراجع‌تان را جایگزین موارد مندرج در آن نمایید.
}

برای بسیاری از مقالات لاتین حتی لازم نیست که مدخل مربوط به آنرا خودتان بنویسید. با جستجوی 
\textbf{نام مقاله + کلمه
\lr{bibtex}}
در اینترنت سایت‌های بسیاری همچون
\lr{ACM} و \lr{ScienceDirect}
را خواهید یافت که مدخل
\lr{bibtex}
مربوط به مقاله شما را دارند و کافیست آنرا به انتهای فایل
\lr{MyReferences.bib}
اضافه کنید.

\subsection{نحوه اجرای \texorpdfstring{\lr{Bib\TeX}}{Bib\TeX}}
پس از قرار دادن مراجع خود، برای ساخت فایل خروجی می‌توانید دستور زیر را (در ترمینال یا از طریق \lr{Texmaker}) اجرا کنید:%
\footnote{فایل \lr{latexmkrc} باید در کنار \lr{main.tex} وجود داشته باشد.}

\singlespacing
\begin{LTR}
	\begin{verbatim}
		latexmk -bibtex -pdf main.tex
	\end{verbatim}
\end{LTR}
\doublespacing
ابزار \lr{latexmk} مراحل مختلف ساخت خروجی لاتک را به طور خودکار و بهینه انجام می‌دهد و هر بار فقط مراحلی را که لازم باشد تکرار می‌کند.
روش دستی‌تر این است که یک بار \lr{XeLaTeX} را روی سند خود اجرا نمایید، سپس \lr{bibtex} و پس از آن هم ۲ بار \lr{XeLaTeX} را. در \lr{TeXMaker} کلید \lr{F11} و در \lr{TeXWorks} هم گزینهٔ \lr{BibTeX} از منوی \lr{Typeset}، \lr{BibTeX} را روی سند شما اجرا می‌کنند.

\section{واژه‌نامه‌ها و فهرست اختصارات}
\gls{Gloss}
یا فرهنگ لغات، مجموعه‌ای از اصطلاحات و تعاریف خاص و فنی است که معمولاً در انتهای یک کتاب می‌آید. چون پایان‌نامه خود یک متن تخصصی بلند محسوب می‌شود، استفاده از فرهنگ لغات در انتهای آن به شدت توصیه می‌شود، خصوصاً که احتمال استفاده از لغات تخصصی لاتین در آن بالاست.
واژه‌نامه‌هایی که در انتهای کتاب‌های انگلیسی می‌آیند معمولاً تک‌زبانه هستند و معنی یک اصطلاح تخصصی در آنها، عمدتاً به صورت یک
\gls{Description}
طولانی آورده می‌شود. اما چون در متون فارسی، آوردن لغات انگلیسی مجاز نیست و باید معادل فارسی آنها وارد شود، جهت رفع ابهام معمولاً واژه‌نامهٔ فارسی به انگلیسی (و برعکس) در انتهای کتاب درج شده و  
\glspl{Description}
در صورت نیاز در متن آورده می‌شوند.

فهرست
\glspl{Acronym}
شامل نمادهای کوتاهی است که اغلب از حروف ابتدایی کلمات یک عبارت طولانی ساخته شده‌اند. با اینکه
\glspl{Acronym}
با حروف (بزرگ) لاتین نوشته می‌شوند، اما چون کوتاهند استفاده از آنها در میان متن فارسی مجاز است. با این حال برای رفع ابهام، عرف است که فهرستی از آنها شامل معنی هر نماد، در کنار دیگر فهرست‌ها در ابتدای متن درج شود.

در این قالب پایان‌نامه، برای ساخت و مدیریت واژه‌نامه و فهرست اختصارات از بستهٔ پیشرفتهٔ
\lr{glossaries}
با موتور واژه‌نامه‌سازی
\lr{xindy}
استفاده می‌شود. تنظیمات بهینهٔ این بسته در فایل
\lr{glossaries-settings.tex}
عبارتند از:
\begin{itemize}
	\item
قبل از درج واژه‌ها در متن، باید مدخل آنها با دستور زیر (ترجیحاً در فایل جدای \lr{words.tex}) تعریف شود:
	\begin{LTR}
	\verb|\newword{Label}{Word}|\{واژه\}\{واژه‌ها\}
	\end{LTR}
	
	\item
قبل از وارد کردن علائم اختصاری در متن، باید مدخل آنها نیز (ترجیحاً در فایل \lr{acronyms.tex}) به صورت زیر تعریف شود:
	\begin{LTR}
	\verb|\newacronym{Label}{Acr}|\{معنی‌اختصار\}
	\end{LTR}

	\item
جهت درج یک علامت اختصاری یا معادل یک واژه تخصصی، کافی است از دستور
	\verb|gls{Label}|
در متن استفاده کنید. دستور
	\verb|glspl{Label}|
نیز برای آوردن معادل یک لغت در حالت جمع ساخته شده است.
	
	\item
هنگام اولین استفاده از یک معادل فارسی یا اختصار در متن، معادل انگلیسی یا معنی آن در پاورقی آورده می‌شود. در صورتی که هر یک از این پیش‌فرض‌ها را دوست ندارید با ویرایش فایل
	\lr{glossaries-settings.tex}
می‌توانید آن را تغییر دهید.

	\item
در انتهای پایان‌نامه با دستور
\verb|\printglossary|
فهرست کلمات استفاده‌شده به ترتیب الفبای فارسی (واژه‌نامه فارسی به انگلیسی) و الفبای انگلیسی (واژه‌نامه انگلیسی به فارسی) درج می‌شود.
\end{itemize}

به عنوان مثال، با مشاهدهٔ کد این نوشته، نحوهٔ درج معادل فارسی
\gls{RandomVariable}
را در متن مشاهده می‌کنید.
در نمایش واژهٔ
\gls{RandomVariable}
برای بار دوم، معادل لاتین در پاورقی نمی‌آید.
در مورد درج علائم اختصاری، مثلاً می‌توان به رابطهٔ
\gls{F}
اشاره کرد.

\section{حاشیه‌نویسی در نسخه پیش‌نویس}
اصلاح و بازبینی چندین و چندبارهٔ یک پایان‌نامه یا مقاله، از معمول‌ترین امور در نگارش آن می‌باشد. فرض کنید دانشجو پایان‌نامه یا مقالهٔ خود را (کامل یا ناقص) نوشته و می‌خواهد نظر استاد راهنما، اعضای آزمایشگاه یا دیگر متخصصین را در مورد آن جویا شود. به جز مشاورهٔ حضوری، تلفنی یا از طریق ایمیل، برای اظهارنظر دقیق بر نوشته، می‌توان از ابزارهای حاشیه‌نویسی در فایل
\lr{PDF}
یا \lr{tex}
نیز استفاده کرد.

یک راهکار مناسب برای حاشیه‌نویسی در فایل \lr{tex}، استفاده از بسته 
\lr{todonotes}
می‌باشد که آقای خلیقی به تازگی امکان استفاده از آن را برای فارسی‌زبانان نیز فراهم آورده‌اند.
بدین منظور، هر جایی که خواستید نکته یا نکاتی را در حاشیه متن یادداشت کنید، کافی است دستور زیر را وارد نمایید:
\begin{latin}
\verb|\todo{NOTE}|
\end{latin}
مثلاً استاد راهنما می‌تواند از دانشجو بخواهد که در بخشی توضیح بیشتری دهد.
\todo{تست تودو}
استاد راهنما یا داور حتی می‌تواند محل پیشنهادی برای درج یک تصویر را نیز به راحتی برای دانشجو مشخص کند.
\missingfigure[figwidth=\textwidth,figcolor=white]{یک تصویر از خروجی الگوریتم 
\ref{alg:RANSAC}
را در اینجا قرار دهید.}
یکی دیگر از امکانات این بسته آن است که می‌توان فهرست نکات را در ابتدای سند داشت. بسته 
\lr{todonotes}
امکانات بسیاری دارد
\todo[fancyline,color=green!30]{تست تودو با خط}
که در راهنمای آن معرفی شده است و با اجرای دستور زیر در خط فرمان می‌توانید آنها را مشاهده کنید:
\begin{latin}	
	\texttt{texdoc todonotes}
\end{latin}	
دقت کنید که توضیحات حاشیه‌ای و فهرست کارهای باقیمانده (نکات)،
\textbf{فقط در نسخه
\gls{Draft}}
قابل دیدن هستند و در نسخه نهایی، نمایش داده نخواهند شد.
برای استفاده از حالت
\gls{Draft}
باید گزینه 
\lr{draft}
به دستور 
\verb|\documentclass|
در ابتدای فایل 
\lr{main.tex}
اضافه شود.
هنگامی‌که سند شما در حالت 
\gls{Draft}
باشد:

\singlespacing
\begin{enumerate}
\item 
هیچ یک از صفحات آغازین پایان‌نامه، تا فهرست مطالب نمایش داده نمی‌شود (به جز صفحه اول).
\item
روی صفحه اول عبارت «پیش‌نویس» به صورت درشت و کم‌رنگ نمایش داده می‌شود.
\item
فهرست نکات درج شده توسط
\lr{todo}،
پس از فهرست اصلی و با عنوان «فهرست کارهای باقیمانده» نمایش داده می‌شود.
\item
شماره صفحاتی که به هر مرجع ارجاع داده شده است در بخش مراجع نمایش داده می‌شود
\footnote{اعمال گزینهٔ
\lr{pagebackref}
برای بستهٔ
\lr{hyperref}.
}.
\end{enumerate}
\doublespacing
هر یک از موارد بالا تا زمانی که نسخه نهایی \پ نیاز نباشد بسیار مورد توجه و مفید واقع می‌شوند.
   	% پیوست سوم: مراجع، واژه‌نامه و حاشیه‌نویسی

% برگرداندن شماره‌بندی صفحات فصول
% \let\chapter\Chapter
\pagenumbering{tartibi} % اول، دوم، ...
%\baselineskip=.75cm

% چاپ واژه‌نامه‌ها و نمایه 
\onehalfspacing
\cleardoublepage
\printglossary
\cleardoublepage
\printindex

\begin{latin}
\baselineskip=.6cm
\latinabstract
\latinTitlePage
\end{latin}
\label{LastPage}

\end{document}
