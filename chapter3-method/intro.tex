هدف اصلی این پایان‌نامه تشخیص زود هنگام زلزله است. برای تشخیص زودهنگان زلزله ابتدا نیاز به تخمینی از مکان وقوع لرزش است و پس از آن باید با کمک روابط تضعیف تخمینی از میزان لرزش در مکان های هدف ارائه شود و در انتها بر اساس آن تخمین هشدارهای مربوط صادر شوند.
\\ 
پس ابتدا روش‌های پیشنهاد شده برای تخمین مکان لرزش توضیح داده شده‌است.
سپس بستر های پیاده‌سازی این الگوریتم‌ها معرفی شد‌‌ه‌است.
قدم بعدی بررسی روابط تضعیف و نحوه تخمین آن‌ها است. در انتهای فصل به بررسی نحوه تخمین میزان لرزش در مکان هدف پرداخته شده‌است.